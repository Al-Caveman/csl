\documentclass{article}
\usepackage{url}
\usepackage{amsmath}
\usepackage{amsfonts}
\usepackage{arabtex}    % to handle arabic
\usepackage{utf8}       % to handle arabic
\setcode{utf8}          % to handle arabic
\author{Caveman\\
email: toraboracaveman@gmail.com\\
website: \url{https://github.com/al-caveman/csl}}
\title{
0.1.1 \<طَبَقَةُ مِقْبَسِ رَجُلِ الكهفِ (طَمِرَك) إصدارة>\\
De Holbewoner Socket Laag (HSL) Versie 1.1.0\\
The Caveman Socket Layer (CSL) Version 1.1.0\\
{\large \<إصلاح الإنحطاط في إس إس إل أو تي إل إس>}\\
{\large Festsetzung der Dekadenz in TLS/SSL}\\
{\large Fixing the Decadence in SSL/TLS}}
\begin{document} \maketitle
\begin{abstract}
    Basically, TLS/SSL sucks. They do too much dance, and PKI and shit. Result?
    Too much hard-to-spot real issues, and you end up needing to trust over
    $300$ CAs, which is probably more than the HTTPS websites that you visit
    per year that you care about. If you think about it, you probably care
    about your email, eBay, Amazon, online banking, etc, which --if you list--
    you will realize that they are less than $300$ many SSL/TLS services that
    you visit. So it is practically more scalable for you to simply white-list
    special keys per ``secure'' service that you use than white-listing $300+$
    CAs. This is funny, cause one of the points of white-listing CAs is that
    you will, supposedly, never need to white-list more cause there are fewer
    CAs than there are secure services that you care about. But it's funny that
    the CAs that you white-list in your OS are more than the secure services
    that you care about. So the CA scalability argument is just tossed right
    there. Total bullshit. Fuck that. Here I fix that by proposing the Caveman
    Socket Layer (CSL) which has two key properties: 1) simple, and 2) works.
    Because of its simplicity, you will be less likely to end up having
    security holes because of weird unpredictable mathematical voodoo. E.g. in
    RSA, if your random sequences, or the random sequences of the other end,
    are shit, then your entire security is shit.  CSL fixes that and more.
    Plus, CSL works just as securely, if not better.
\end{abstract}

\tableofcontents

\section{Overview}
We got a client, and a server that's identified as $S$  and is listening on
port $P$. The identifier $S$ could be the domain name, IP address, etc. The
client chooses how it identifies the server based on client's policies. Client
wishes to communicate with server. Procedure goes as follows:
\begin{enumerate}
    \item Server, when initializing the CSL service, needs to generate a good
    random sequence. Perhaps by using \texttt{/dev/random}, or whatever other
    method. We don't care about that. All we need, get enough random bits. How
    many bits are enough? This to be decided by the DevOps at hand. But for
    now, we get a random sequence, which we call $R_s$ for now.
    \item Then, after we have $R_s$ figured out, we can start the CSL service
    and have it listen on port $P$.
    \item If there is no CSL caching entry in the client about $S$:
        \begin{enumerate}
            \item Client warns user ``\emph{$S$ is a new server, wanna learn
            its CSL fingerprint?}''. Unless user chooses ``\emph{yes}'',
            process will terminate right here.
            \item Client uses $S$ to identify how to communicate with the
            server. E.g.  if $S$ is the domain name, client performs a DNS
            lookup to get the IP address of the server.
            \item Client opens TCP (or UDP? Whatever) connection to port $P$.
            \item Client and server do the Diffie-Hellman (DH) dance to agree
            on a an encryption key $k$. From now and on, all exchanged messages
            by the client or the server are encrypted by using some symmetric
            encryption function $e$ (e.g. $e$ could be some variant of AES?)
            using key $k$:
                \begin{enumerate}
                    \item Client generates its own random sequence $R_c$.
                    \item Client sends the following Type-Length-Value (TLV)
                    tuple: $(\text{init}, n, R_c)$ to server via this
                    connection, where ``init'' denotes that this is the
                    initialization phase, $n$ denotes size of $R_c$ probably in
                    octets, or whatever other more convenient unit.
                    \item Server reads $R_c$, and sends the TLV
                    $\big(\text{initre}, n, h(R_s + R_c)\big)$, where $h$ is
                    some lovely hashing function that tickles your fancy (e.g.
                    SHA3), and $+$ denotes concatenation.
                \end{enumerate}
            \item Client then locally caches the following information:
                \begin{itemize}
                    \item $S$.
                    \item $R_c$.
                    \item $h(R_s + R_c)$.
                \end{itemize}
            \item Client and server then proceed communicating with each other
            using a stream cipher using $h(R_s + R_c)$ as the
            encryption/decyption pre-shared key.

            \item Upon connection termination, while still being in the
            encryption channel, the following is done:
                \begin{enumerate}
                    \item Client generates a new random sequence, $R_d$.
                    \item Client sends TLV tuple $(\text{init}, n, R_d)$ to
                    server.
                    \item Server responds by $h(R_s + R_d)$.
                    \item Client updates its local cache regarding server $S$
                    as follows:
                        \begin{itemize}
                            \item $S$.
                            \item $R_c := R_d$.
                            \item $h(R_s + R_c) := h(R_s + R_d)$.
                        \end{itemize}
                        where $:=$ is assignment.
                \end{enumerate}
        \end{enumerate}
    If there is a CSL caching entry in the client about $S$:
    \begin{enumerate}
        \item Client connects to $S$ at port $P$.
        \item Client sends TLV tuple $(\text{auth}, n, R_c)$ to server.
        \item Client then sends the following TLV, encrypted using the
        stream cipher described above and the cached $R_c$
        and $h(R_s + R_c)$ values: $(\text{init}, n, R_c)$.
        \item Within this stream cipher encryption channel, the
        server responds: $\hat h(R_s + R_c)$
        \item If $\hat h(R_s + R_c) \ne h(R_s + R_c)$:
            \begin{enumerate}
                \item Warn user and terminate connection. Perhaps maybe ask
                user if he/she wants to re-negotiate a new server fingerprint?
                This is left to the user interface side, but for now, we can
                say at least "warn".
            \end{enumerate}
        Else, i.e. if $\hat h(R_s + R_c) = h(R_s + R_c)$:
            \begin{enumerate}
                \item Client generates a new random sequence $R_e$.
                \item Client sends $(\text{init}, n, R_e)$ to server.
                \item Server replies $h(R_s + R_e)$.
                \item Client updates cache.
                \item Client and server communicate using stream cipher with
                key $h(R_s+R_e)$.
                \item Upon connection termination, the client updates the cache
                entry.
            \end{enumerate}
        \textbf{Note 4:} note how the client and server avoid the use of DH for
        when there is a CSL cache entry. DH was only used when there was no CSL
        cache entry. This should bring more speed gains.
    \end{enumerate}
\end{enumerate}

\textbf{Note about SNI:} everything happens as usual, except for sending the
TLV $(\text{fwd}, n, \text{www.domain.com})$ first.

\section{Messages Format}
All CSL data are communicated, using big-endian byte order per field, and
messages that are formated as the following tuples:
\begin{itemize}
    \item TV: 
        \begin{tabular}{|l|l|}\hline
            $t$ (8 bits) & $d$ (length defined per $t$)\\\hline
        \end{tabular}
    \item TLV:
        \begin{tabular}{|l|l|l|}\hline
            $t$ (8 bits) & $n$ (32 bits) & $d$ (length bits)\\\hline
        \end{tabular}.
\end{itemize}

\section{Message Types}
\subsection{Version (TV)}
\begin{itemize}
    \item $t = 1$ (in decimal).
    \item $d = 0$ (8 bits long, in decimal).
    \item Overview: to advertise client's version. 
    \item Prerequisite: none.
    \item Usage:
        \begin{enumerate}
            \item Client sends the TLV with $d=0$ to denote that this is
            version 0.
            \item When using version $0$, the hashing function is SHA3-256, and
            $15$ is the DH group as per RFC3626.
            \item Salsa20 in CTR mode is used for the stream cipher.
        \end{enumerate}
\end{itemize}

\subsection{Bye (TV)}
\begin{itemize}
    \item $t = 2$ (in decimal).
    \item $d \in \{0,1,\ldots,255\}$ (8 bits long, in decimal).
    \item Overview: to define cause of termination. 
    \item Prerequisite: none.
    \item Usage:
        \begin{enumerate}
            \item Terminating peer sets $d=0$ if cause of termination cause is
            due to unsupported versions. Note: this only indicates the cause of
            termination, while the termination itself is carried out by the
            transport layer.
        \end{enumerate}
\end{itemize}

\subsection{Initial Encryption Channel (TLV)}
\begin{itemize}
    \item $t = 3$ (in decimal).
    \item $n \in \{1,2,\ldots,2^{32}\}$ (in decimal).
    \item $d \in \{0,1,\ldots,255\}^n$ (in decimal).
    \item Overview: used to signal that the sending peer is interested in
    finding a shared key based on $d$ and DH, and then to start communicating
    over a symmetric encrypted channel that uses the shared key. 
    \item Prerequisite: none.
    \item Usage:
        \begin{enumerate}
            \item Client starts the DH process locally to calculate the number
            that is to be publicly shared with the server. This number is sent
            as $d_c$ along with its size $n_c$ using this message type.
            \item When the server receives this message, it, too, starts its
            own DH process to send its own publicly-shared number as $d_s$
            along with its length $n_s$ to the client, by using the same
            message type. The server then computes the shared key $k_s$. Now,
            any data the server receives from the client must be encrypted by
            $k_s$, and any data the server sends to the client must be
            encrypted by $k_s$.
            \item When the client receives the number $d$ from the server, the
            client then also computes the shared key $k_c$.  Now, any data the
            client receives from the server must be encrypted by $k_c$, and any
            data the client sends to the server must be encrypted by $k_c$.
            Thank to the way DH works, we know that $k_s = k_c$, and a
            symmetric encryption channel can start.
        \end{enumerate}
\end{itemize}

\subsection{Encrypted Signalling Message (TLV)}
\begin{itemize}
    \item $t = 4$ (in decimal).
    \item $n \in \{1,2,\ldots,2^{32}\}$ (in decimal).
    \item $d \in \{0,1,\ldots,255\}^n$ (in decimal).
    \item Overview: send some encrypted data by using the latest key that was
    agreed upon. The receiver must decrypt it, and then parse the decrypted
    form as a CSL TLV.
    \item Prerequisite: encryption shared key must be known.
    \item Usage:
        \begin{enumerate}
            \item The sender sets $d$ to be some encrypted message, and $n$ to
            be the length of $d$.
            \item The sender must encrypt $d$ by the latest agreed encryption
            shared key.
            \item The receiver must decrypt $d$ by the latest agreed encryption
            shared key.
            \item The receiver must then parse the decrypted $d$ as a CSL TLV,
            and act upon it accordingly.
        \end{enumerate}
\end{itemize}

\subsection{Encrypted Data Message (TLV)}
\begin{itemize}
    \item $t = 5$ (in decimal).
    \item $n \in \{1,2,\ldots,2^{32}\}$ (in decimal).
    \item $d \in \{0,1,\ldots,255\}^n$ (in decimal).
    \item Overview: send some encrypted data by using the latest key that was
    agreed upon. The receiver must decrypt it, and then forward the decrypted
    data as it is to the upper-layer application.
    \item Prerequisite: encryption shared key must be known.
    \item Usage:
        \begin{enumerate}
            \item The sender sets $d$ to be some encrypted message, and $n$ to
            be the length of $d$.
            \item The sender must encrypt $d$ by the latest agreed encryption
            shared key.
            \item The receiver must decrypt $d$ by the latest agreed encryption
            shared key.
            \item The receiver must then forward the decrypted data to the
            upper-layer application.
        \end{enumerate}
\end{itemize}

\subsection{Request Fingerprint (TLV)}
\begin{itemize}
    \item $t = 6$ (in decimal).
    \item $n \in \{1,2,\ldots,2^{32}\}$ (in decimal).
    \item $d \in \{0,1,\ldots,255\}^n$ (in decimal).
    \item Overview: client asks the server to send a server fingerprint based
    on client's random sequence $R_c$.
    \item Prerequisite: this message must be encapsulated in a ``Encrypted
    Signalling Message''.
    \item Usage:
        \begin{enumerate}
            \item Client sends TLV to the server such that $d = R_c$, and $n$
            to be the length of $R_c$ in octets.
            \item Client must not send any other messages to the server.
            \item Server responds by sending ``Fingerprint'` message.
            \item Client will act accordingly as described in
            the``Fingerprint'' message section.
        \end{enumerate}
\end{itemize}

\subsection{Fingerprint (TLV)}
\begin{itemize}
    \item $t = 7$ (in decimal).
    \item $n \in \{1,2,\ldots,2^{32}\}$ (in decimal).
    \item $d \in \{0,1,\ldots,255\}^n$ (in decimal).
    \item Overview: server responds to client by sending its fingerprint that
    is based on client's random sequence $R_c$. This is sent after the server
    receives a ``Request Fingerprint'' message.
    \item Prerequisite: a ``Request Fingerprint'' message, that hasn't been
    addressed, must have been received, and this message must be encapsulated
    in a ``Encrypted Signalling Message''.
    \item Usage:
        \begin{enumerate}
            \item Server sends to the client a TLV message with $d=h(R_s+R_c)$,
            and $n$ is the length of that.
            \item Server must assume that $d$ is the encryption shared key of
            any subsequently received and sent messages. I.e. all
            outgoing/incoming messages will be encrypted/decrypted by $d$,
            respectively.
            \item When client receives this $d$, it must assume that $d$ is the
            new shared encryption key. Client is then able to sending encrypted
            messages to the server.
        \end{enumerate}
\end{itemize}

\subsection{Server Name Indication (TLV)}
\begin{itemize}
    \item $t = 8$ (in decimal).
    \item $n \in \{1,2,\ldots,2^{32}\}$ (in decimal).
    \item $d \in \{0,1,\ldots,255\}^n$ (in decimal).
    \item Overview: client specifies that all subsequent messages are forwarded
    to the server that is named/identified in $d$.
    \item Prerequisite: none.
    \item Usage:
        \begin{enumerate}
            \item Receiving server starts relaying all messages from the
            sending client to the server that is identified by $d$. Likewise,
            responses from $d$ are relayed back to the sending client.
        \end{enumerate}
\end{itemize}

\subsection{Message Size (TV)}
\begin{itemize}
    \item $t = 9$ (in decimal).
    \item $d \in \{1,2,\ldots,2^{32}\}$ (32 bits, in decimal).
    \item Overview: client sets the message length.
    \item Prerequisite: none.
    \item Usage:
        \begin{enumerate}
            \item Client sets $d$ to the desired length of messages in octets.
            \item Server must then ensure that all message values that it sends
            to be padded by zeros (if necessary), such that the total length of
            the message values equals $d$. Note, this does not affect the $n$.
            I.e. $n$ is set to the true message length before any pads are
            postfixed to the message value.
            \item If server does not honour the required padding to satisfy
            target message length, the client must notify the user of such
            violation. Note: the client can always apply padding to its own
            messages when sending messages to a server.
        \end{enumerate}
\end{itemize}

\section{Discussion}
If you think about it, you will realize that this is better than SSL/TLS for
practical purposes for whatever SSL/TLS is used for today. You can ask me
questions via email or IRC. You can get IRC info via email.

Minor refinements are needed about the message formats, or
about how new CSL server fingerprints are negotiated. But the general idea
remains the same. If you think deep enough, you'd realize that this is superior
to SSL/TLS as it avoids the stupid PKI architecture, much faster, and much
simpler algorithm that significantly minimizes the chance of leaving
mathematical loopholes undetected.

\end{document}
